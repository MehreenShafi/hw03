%CS-113 S18 HW-3
%Released: 16-Feb-2018
%Deadline: 2-March-2018 7.00 pm
%Authors: Abdullah Zafar, Waqar Saleem.


\documentclass[addpoints]{exam}

% Header and footer.
\pagestyle{headandfoot}
\runningheadrule
\runningfootrule
\runningheader{CS 113 Discrete Mathematics}{Homework II}{Spring 2018}
\runningfooter{}{Page \thepage\ of \numpages}{}
\firstpageheader{}{}{}

\boxedpoints
\printanswers
\usepackage[table]{xcolor}
\usepackage{amsfonts,graphicx,amsmath,hyperref}

\title{Habib University\\CS-113 Discrete Mathematics\\Spring 2018\\HW 3}
\author{ms04030}  % replace with your ID, e.g. oy02945
\date{Due: 19h, 2nd March, 2018}


\begin{document}
\maketitle

\begin{questions}



\question
All sets carry data, but how much information can be extracted from it? Consider a simple model on a set $A$, in which each relation encodes 1 unit of information. We define the ``Information Potential" of a set as the sum of information units (or the number of distinct relations) that can be generated from the set. In the questions that follow, you may assume all relations to be binary.

\begin{parts}
  \part Consider $A$ to be the set of $n$ distinct facts. What is the information potential of this set?
  
  \begin{solution}
    % Write your solution here
    Information potential of A  is 
    
    If $|A| = n$
    \\ $ S = A \times A$
   \\ $|P(S)| = 2^{n^2}$
    
    
  \end{solution}
  
  \part Reflexive pairs of the form $(fact\;x, fact\;x)$ are considered redundant in our model. What is the information potential of the ``non-redundant" set, that is, the set without reflexive relations? 
  
  \begin{solution}
    % Write your solution here
    The number of reflexive relations in a set of size $n$ is $2^{n^2 - n}$
    
    So the cardinailty of a set $H$ of relations without the reflexive elements would be,
    
    $H =$ $ 2^{n^2} - 2^{n^2-n} $
    
    
  \end{solution}

  \part Anti-symmetric relations that follow the rule $(fact\;x,fact\;y)\; \land (fact\;y,fact\;x) \rightarrow fact\;x = fact\;y$ are of special interest to our model. Such pairs, as in the aforementioned antecedent, can be used to express ordered relationships between facts. What is the combined Information Potential of anti-symmetric relations on the non-redundant set? 
  \begin{solution}
    % Write your solution here
    
    There are three possibilities for an anti-symmetric set,
    \begin{itemize}
        \item  Only $(x,y)$ is present
        \item  Only  $(y,x)$ is present
        \item  Neither are present
    \end{itemize}

    Thus, the condition of anti-symmetric set becomes,
    
    $|A| = $ $ {}^n {C_3} $
    
    $|A| = $ $ 3^\frac{{n^2} - n}{2}$
    
    
    Thus without reflexive relations the expression becomes,
    
    $ |A|=$ $ (2^n-1 ) . (3^\frac{{{n^2} - n}}{2})$
    
    
    
    
    
  \end{solution}
  
  \part There are many ways to describe a relation in natural language. For example, a relation described as $``x<y"$ over the set $\{1,2\}$ may also be described as $``x+1=y"$. Specifically, two descriptions that produce the same relation are considered ``isomorphs" of one another in our model. There may be any number of isomorphs for a given relation. Given $2^{n^2+1}$ descriptions of relations, how many isomorphs exist? (Give your answer as a range)
  
  \begin{solution}
    % Write your solution here
    
    The total number of isomorphs found in a number of $n$ relations would be $n$.
    Similarly, the maxiumum number of isopmorphs for ${2^{n^{2}+1}}$ would be ${2^{n^{2}+1}}$
    
    The least amount of isomorphs possible are, 
    the element plus it's isomorphs (which are the total number of relations)
    
    $ 2^{n^2} + 1 $
    
    Thus, the range is
    
    
    $ 2^{n^2} + 1 $ $\leq$ $|i|$  $\leq$ ${2^{n^{2}+1}}$
    
    
  \end{solution}

\end{parts}

\question Let $R$ be a relation from $A$ to $B$. Then the inverse of $R$, written $R^{-1}$, is a relation from $B$ to $A$ defined by $R^{-1} = \{(y,x) \in B \times A \:|\: (x,y) \in R\}$. Prove that $R$ is symmetric if $R = R^{-1}$.

  \begin{solution}
    % Write your solution here
    
    $\implies$ $R$ and $R^{-1}$ are equal    
     \\ $(x,y) \in R$ $\rightarrow (y,x) \in R^{-1}$    
    \\If $R = R^{-1}$ then this means that , for every element
    that is in $R$, its inverse is in $R^{-1}$, which is also in are because they are equal.
    
    Thus, this proves that $R$ is symmetric
    
    
    
  \end{solution}

\question Let $R$ and $S$ be relations on a set $A$. Assuming $A$ has at least 3 elements, state whether each of the following statements is true or false, providing a brief explanation if true, or a counterexample if false:
\begin{parts}
\part If $R$ and $S$ are reflexive, then $R \cup S$ is reflexive.
\part If $R$ and $S$ are anti-symmetric, then $R \circ S$ is anti-symmetric.
\part If $R$ and $S$ are symmetric, then $R \cap S$ is symmetric.
\part If $R$ is reflexive, then $R \cap R^{-1}$ is not empty. 
\part If $R$ is transitive, then $R^{-1}$ is transitive.
\end{parts}


  \begin{solution}
    % Write your solution here
    
    \begin{parts}
    \part True.
    If both are reflexive, then their union will also contain the reflexive relations of all elements of R as well as S, thus it will also be reflexive.
    
    \part False. 
    $\{(a,b) \in R$ $\wedge$
    $(b,c) \in S$
    $\rightarrow (a,c) \in R \circ S \}$
    
    If R and S are anti-symmetric, then $R \circ S $ becomes symmetric
    \\$R = \{(2,5),(3,4)\} $
    \\$S = \{(2,3),(4,5)\}$
    \\$R \circ S = \{(2,4),(4,2)\}$
    
    \part True. If both are symmetric, their intersection will also include the symmetric elements thus it will also be symmetric.
    
    \part True. If $(x,y) \in R$ $\rightarrow (y,x) \in R^{-1}$
    If R is reflexive $x=y$
    Thus, there is at least one element in $R$ which is also in $R^-1$
    
    \part True. 
    If R is transitive,
    $\{( (a,b) \in R \wedge (b,c) \in R) \rightarrow ((a,c) \in R)\}$
    It's inverse would be
    $\{( (c,b) \in R^{-1} \wedge (b,a) \in R^{-1}) \rightarrow ((c,a) \in R^{-1})\}$
    which is also transitive.
    
    \end{parts}
  \end{solution}
  
\question Let $R$ be a relation on $A$. Prove that the digraph representation of $R$ has a path of length $n$ from $a$ to $b$ iff $(a, b) \in R^n$.

  \begin{solution}
    % Write your solution here
    
    This can be proven through mathematical induction.
    
    Let us check that the statement is true for $n=1$, i.e.,
    
    There is a path from $a$ to $b$ in $R^1$, which means that
    $\{(a,b) \in R^1\}$
    
    As this is proven true by definition, we now assume that it is also true for $n = k$
    
    Take $n = k+1$
\\So, for this relation there is a path from $a$ to $b$ if $(a,c) \in R^{k+1} $ $\wedge$ $(c,b) \in R^{k+1} $ $\rightarrow (a,b) \in R^{k+1}$ 


Thus, if it is true for $n = k+1$ then it is true for $n= k$ which means that it is true for all $n$

\textbf{Hence Proved!}
    
    
  \end{solution}

\question
    Let $R$ be a relation on a set $A$. We define

    $\rho (R) = R \cup \{(a, a) | a \in A\}$ \\ 
    $\phi (R) = R \cup R^{-1}$ \\
    $\tau (R) = \cup \{ R^n | n = 1,2,3,...\}$
    
    Show that $\tau (\phi (\rho (R)))$ is an equivalence relation containing $R$.
    
      \begin{solution}
    % Write your solution here
    
    For a relation to be equivalent, it must fulfill three conditions, i.e it must be a
    
    \begin{itemize}
    
    \item Reflexive Relation
    
    \item Symmetric Relation
    
    \item Transitive Relation
    
    
    
    
    \end{itemize}
    
    To start with, $\rho (R)$ is a reflexive relation which implies that
    
     $\phi(\rho(R))$ and $\tau(\phi(\rho(R)))$ are also reflexive, due to their domain being reflexive.
     
      If $(a,b) \in R \rightarrow (b,a) \in R^{-1}$ and as $\phi (R)$ is a union of both, it contains both relations and thus is a symmetric set which makes
     $\tau(\phi(\rho(R)))$ also symmetric.
    
    
    A relation can be proved transitive by the following theorem,
    
    $R \subseteq R^n$ \\
    
    where $R^n = R^1 \cup R^2 \cup R^3... R^n$
    
    
    By definiton,
     $\tau (R) = \cup \{ R^n | n = 1,2,3,...\}$ that is $R \subseteq R^n$ 
     
    Thus,  $\tau(\phi(\rho(R)))$ is transitive.
    
    
    As all three conditions have been met, we can conclude that  $\tau(\phi(\rho(R)))$ is an equivalent relation.
     
    \textbf{Hence Proved!}
    
    
  \end{solution}


\end{questions}

\end{document}
